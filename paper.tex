\documentclass[onecolumn]{aastex63}
\usepackage{tikz}
\usetikzlibrary{shapes,arrows}

\usepackage{subfiles}

%\newcommand{\TODO}[1]{{\it \color{red} (#1)}}
\newcommand{\TODO}[1]{{\bf (#1)}}

\begin{document}
\shortauthors{People}
\shorttitle{The DESI spectroscopic pipeline}
\title{
Fiber assignment in the Dark Energy Spectroscopic Instrument 
}

\author{
  Some people
}


\begin{abstract}
The fiber assignment process assigns a robotic fiber positioner to a target.\end{abstract}

\keywords{cosmology}


\tableofcontents


\section{Introduction}


\section{DESI in a Nutshell}
\label{sec:Introduction}
\TODO{what is DESI project and instrument ; important performance aspects of the pipeline for DESI
  inheritance from SDSS (design but not code), Requirements}
In \cite{Bolton2012}, bla ... 

\subsection{General Overview}


\subsection{Hardware constraints}

Positioner shape.

Focal plane shape.


\subsection{Spectroscopic Constraints}

We need a minimum number of sky and std star fibers.

\section{Inputs, Outputs and Algorithms}

\subsection{Inputs}

Hardware State.

Input target catalog.

Tiles to be observed.

\subsection{Outputs}

\subsection{Algorithms}

\section{Numerical Experiments}

\subsection{Random points, single class targets}

\subsection{Random points, multi-class targets}

\subsection{Clustered points, multi-class targets}

\section{Metrics}

\subsection{Running Times}

\subsection{Completeness}

\section{Results}

\section{Conclusions}

\include{acknowledgments}

\bibliographystyle{aasjournal}
\bibliography{biblio}


\end{document}
